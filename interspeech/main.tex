% *==================================================================================*
% *                     Review vs. Camera-Ready settings                             *
% *==================================================================================*
%
% REVIEW: Use the following command for submitting the paper (double-blind,
% for review):
% \documentclass{Interspeech}
%
% CAMERA-READY: Use the following command for the camera-ready version, one
% affiliation per line:
\documentclass[cameraready]{Interspeech}
% *==================================================================================*


% **************************************
% *                                    *
% *      STOP !   DO NOT DELETE !      *
% *          READ THIS FIRST           *
% *                                    *
% * This template also includes        *
% * important INSTRUCTIONS that you    *
% * must follow when preparing your    *
% * paper. Read it BEFORE replacing    *
% * the content with your own work.    *
% **************************************

%==================================================================================
% Title
% Must exactly match the title entered into the paper submission system
\title{Paper Instructions and Template for Interspeech 2026}

%==================================================================================
% Authors
% The order of authors here must exactly match the order entered into the paper submission system
% Note that the COMPLETE list of authors MUST be entered into the paper submission system at the outset, including when submitting your manuscript for double-blind review
% The ORCID number is still optional but will become mandatory in the future years. It is strongly encouraged to get an ORCID for each cu-author.
% Middle names, including initials, must be included in the first name
\author[affiliation={1}, orcid=0000-0000-0000-0000, equalcontribution]{FirstNameA}{LastNameA}
\author[affiliation={2,3}, orcid=0000-0000-0000-1111, equalcontribution, correspondingauthor]{FirstNameB InitialB}{LastNameB}
\author[affiliation={1,3}]{FirstNameC}{LastNameC}
% The maximum number of authors in the author list is 20. If the number of contributing authors is more than this, they should be listed in a footnote or the acknowledgement section.

%==================================================================================
% Affiliations

\address{
    $^1$ Address Affiliation 1, Country Affiliation 1 \\
    $^2$ Address Affiliation 2, Country Affiliation 2 \\
    $^3$ Address Affiliation 3, Country Affiliation 3
}

%==================================================================================
% Emails
\email{first@university.edu, second@companyA.com, third@companyB.ai}

%==================================================================================
% Keywords
\keywords{speech recognition, human-computer interaction, computational paralinguistics}

\newcommand{\blue}[1]{\textcolor{blue}{#1}}
\newcommand{\red}[1]{\textcolor{red}{#1}}
\usepackage{comment}

%==================================================================================
% Content

\begin{document}

\maketitle

% the abstract here must exactly match the abstract entered into the paper submission system
\begin{abstract}
    % 1000 characters. ASCII characters only. No citations.
    Manuscripts submitted to Interspeech 2026 must use this document as both an instruction set and as a template. Do not use a past paper as a template. Always start from a fresh copy, and read it all before replacing the content with your own. The main changes with respect to previous years' instructions are \blue{highlighted in blue}.

    Before submitting, check that your manuscript conforms to this template. If it does not, it may be rejected. Do not be tempted to adjust the format! Instead, edit your content to fit the allowed space. \blue{The maximum number of manuscript pages is 6 for regular papers and 10 for long papers. For regular papers, pages 5 and 6, and for long papers, pages 9 and 10, are reserved exclusively for acknowledgments, disclosures of the use of generative AI tools, and references, which may begin on an earlier page if space permits.}

    The abstract is limited to 1000 characters. The one in your manuscript and the one entered in the submission form must be identical. \blue{In the submission form, no \LaTeX{} code is allowed.} Do not use citations in the abstract: the abstract booklet will not include a bibliography. Index terms appear immediately below the abstract.
\end{abstract}



\section{Introduction}

Templates are provided on the conference website for Microsoft Word\textregistered, and \LaTeX. We strongly recommend \LaTeX\xspace
which can be used conveniently in a web browser on \url{https://overleaf.com} where this template is available in the Template Gallery.

\subsection{General guidelines}

Authors are encouraged to describe how their work relates to prior work by themselves and by others, and to make clear statements about the novelty of their work. This may require careful wording to avoid unmasking the identity of the authors in the version submitted for review (guidance in Section \ref{section:doubleblind}). All submissions must be compliant with the \textbf{ISCA Code of Ethics for Authors}, the \textbf{Pre-prints Policy}, and the \textbf{Conference Policy}. These can be found on the conference website under the ``For authors'' section.

\subsubsection{Originality}

All papers submitted to Interspeech 2026 must be original contributions that are not currently submitted to any other conference, workshop, or journal, nor will be submitted to any other conference, workshop, or journal during the review process of Interspeech 2026. Cross-checks with submissions from other conferences will be carried out to enforce this rule.

\subsubsection{Generative AI tools}

All (co-)authors must be responsible and accountable for the work and content of the paper, and they must consent to its submission. Generative AI tools cannot be a co-author of the paper. They can be used for editing and polishing manuscripts, but should not be used for producing a significant part of the manuscript.
\blue{The use of generative AI tools should be acknowledged in a subsection titled Generative AI Use Disclosure between the Acknowledgments of the References.}






\subsubsection{\blue{List of authors}}

{\color{blue}

When preparing the author list, please observe the following guidelines:
\begin{itemize}
\item Clearly list all authors who have made a significant contribution to the work.
\item The maximum number of authors in the author list is 20. If there are more than 20 contributing authors, the additional authors should be acknowledged in a footnote or in the acknowledgments section.
\item The use of ORCID identifiers for authors is optional but strongly encouraged. It will become mandatory in the future.
\item Submissions must be anonymous: author names and affiliations must not appear in the manuscript during review. However, the full list of authors (and affiliations) must be entered in the submission platform when submitting the paper for review.
\item During the development phase of the paper, authors may keep the full (non-anonymized) list of authors and affiliations in their working copy to verify layout and space constraints. Anonymization should be applied at the final stage before generating the version for submission.
\end{itemize}
}




\subsubsection{Conference theme}

\blue{The theme for Interspeech 2026, "Speaking Together," reflects our dedication to the celebration of all voices. This conference aims to showcase groundbreaking work in under-resourced languages, atypical speech and initiatives to improve access to speech technology for all speakers.}

\subsubsection{Participation in reviewing}

Interspeech uses peer review to evaluate submitted manuscripts. At least one co-author of each paper should be volunteering to review for the Interspeech conference. If you are eligible, but not yet a reviewer for ISCA conferences, you should sign up at

\begin{center}
    \url{https://isca-speech.org/Reviewing}
\end{center}
using the same email address as in your CMT account.
The eligibility criteria can be located through the above link. Papers for which no author qualifies as ISCA reviewer do not have to follow this guideline. During submission, authors should indicate the email of any co-authors who should be added to the reviewer database.

\subsubsection{Accessibility recommendations}

\blue{Authors are encouraged to have their papers fulfill specific requirements relating to \textbf{figures} (see Section \ref{sec:figures}) and \textbf{tables} (see Section \ref{sec:tables}).}


\subsubsection{Scientific reporting check list}

The following checklist will be part of the submission form and includes a list of guidelines we expect Interspeech 2026 papers to follow. A compact version of the checklist is available under ``For authors'' section. Not every point in the check list will be applicable to every paper. Ideally, all guidelines that do apply to a certain paper should be satisfied. Nevertheless, not satisfying a guideline that applies to your paper is not necessarily grounds for rejection.

\begin{enumerate}

\item Claims and limitations - for all papers
\begin{itemize}

\item The paper clearly states what claims are being investigated, and why.
\item The novelties are clearly explained.
\item The main claims made in the abstract and introduction accurately reflect the paper’s contributions and scope.
\item The limitations of your work are described.
\item The relevant assumptions made in your work are stated in the paper.
\end{itemize}

\item If models are used, the paper includes information about the following:
\begin{itemize}
    \item Justification for each selected model, algorithm, or architecture.
    \item Sufficient details of the model (architecture), including reference(s) to prior literature.
\end{itemize}

\item If data sets are used, the paper includes information about the following:
\begin{itemize}
\item Relevant details such as languages, audio duration distribution, number of examples, and label distributions, or a reference where that information can be found.
\item Details of train/validation (development)/test splits. Ideally, the lists should be released with the supplementary material if not already publicly available.
\item Explanation of all pre-processing steps, including any criteria used to exclude samples, if applicable.
\item Reference(s) to all data set(s) drawn from the existing literature.
\item For newly collected data, a complete description of the data collection process, such as subjects, instructions to annotators and methods for quality control and a statement on whether ethics approval was necessary.
\end{itemize}

\item If using non-public data sets or pre-trained models:
\begin{itemize}
\item The paper includes a discussion on the reason/s for using non-public data sets.
\item Full details of the dataset are included in the paper to enable comparison to similar data sets and tasks.
\item For papers that use publicly available, pre-trained (e.g. foundational or self-supervised) models, link to the model is provided. For papers that use non-public pre-trained models, description of the data (including quantity) used for training that model is provided.
\end{itemize}


\item If reporting experimental results, the paper includes:
\begin{itemize}
\item An explanation of evaluation metrics used.
\item An explanation of how models are initialized, if applicable.
\item Some measure of statistical significance of the reported gains or confidence intervals.
\item A description of the computing infrastructure used and the average runtime for each model or algorithm (e.g. training, inference etc).
\item The number of parameters in each model.
\end{itemize}

\item If hyperparameter search (including choice of architecture or features and any other development decision) was done, the paper includes:
\begin{itemize}
\item Final results on a held-out evaluation set not used for hyperparameter tuning.
\item Hyperparameter configurations for best-performing models.
\item The method for choosing hyperparameter values to explore, the range of the hyperparameter values, and the criterion used to select among them.
\end{itemize}

\item If source code is used:
\begin{itemize}
\item The code is or will be made publicly available and/or sufficient details are included in the paper for the work to be reproduced.
\item For publicly available software, the corresponding version numbers and links and/or references to the software.
\end{itemize}

\end{enumerate}

For any paper that uses machine learning, we encourage authors to provide statistical significance tests and follow well-established processes for evaluating any statistical or machine learning model. \blue{McNemar’s test \cite{mcnemar1947} is one of the most popular tests used in the literature to report statistical significance (e.g., \cite{fiscus1997post, chollet1995evaluation, williams2008exploiting, busch2025multilingual, lin2025voice, lee2025performance}). Other examples include ANOVA \cite{allbert2025evaluating}, adding error bars~\cite{miller2024addingerrorbarsevals}, and bootstrapping approaches~\cite{ferrer2024goodpracticesevaluationmachine, url2025}.}

\subsection{Double-blind review}
\label{section:doubleblind}

Interspeech~2026 uses double-blind review to support the integrity of the review process.

\subsubsection{Version submitted for review}

The manuscript submitted for review must not include any information that might reveal the authors' identities or affiliations. This also applies to the metadata in the submitted PDF file (guidance in Section \ref{section:pdf_sanitise}), uploaded multimedia and online material (guidance in Section \ref{section:multimedia}).

Take particular care to cite your own work in a way that does not reveal that you are also the author of that work. For example, do not use constructions like ``In previous work [23], we showed that \ldots''. Instead use something like ``Jones et al. [23] showed that \ldots''.

Acknowledgements should not be included in the version submitted for review.

{\color{blue}
For \LaTeX{} users, make sure to activate the review mode of the template by uncommenting the line \verb|\documentclass{Interspeech}| and commenting the line \verb|\documentclass[cameraready]{Interspeech}| at the beginning of the template.}

Papers that reveal the identity of the authors will be rejected without review.
Note that the full list of authors must still be provided in the online submission system, since this is necessary for detecting conflicts of interest.

\subsubsection{Camera-ready version}

Authors should include names and affiliations in the final version of the manuscript, for publication.

{\color{blue}
For \LaTeX{} users, make sure to activate the camera-ready mode of the template by commenting the line \verb|\documentclass{Interspeech}| and uncommenting the line \verb|\documentclass[cameraready]{Interspeech}| at the beginning of the template.
}

Include the country as part of each affiliation. Do not use company logos anywhere in the manuscript, including in affiliations and Acknowledgements. After acceptance, authors may of course reveal their identity in other ways, including: adjusting the wording around self-citations; adding further acknowledgements; updating multimedia and online material. Acknowledgements, if any, should be added in the camera-ready version.

\subsubsection{Pre-prints}
\label{section:preprints}

\blue{Please visit the submission policy under the \textit{For Authors} section of the Interspeech 2026 website (\url{https://interspeech2026.org/}) for a detailed explanation of the policy on pre-prints for this Interspeech. Interspeech no longer enforces an anonymity period for submissions published in archive repositories.  While uploading a version online is permitted, your official submission to Interspeech must not contain any author-identifying information.}



\section{Format}

\subsection{Layout}

Authors should observe the following specification for page layout by using the provided template. Do not modify the template layout! Do not reduce the line spacing!

\subsubsection{Page layout}

\begin{itemize}
\item Paper size must be DIN A4.
\item Two columns are used except for the title section and for large figures or tables that may need a full page width.
\item Left and right margin are \SI{20}{\milli\metre} each.
\item Column width is \SI{80}{\milli\metre}.
\item Spacing between columns is \SI{10}{\milli\metre}.
\item Top margin is \SI{25}{\milli\metre} (except for the first page which is \SI{30}{\milli\metre} to the title top).
\item Bottom margin is \SI{35}{\milli\metre}.
\item Text height (without headers and footers) is maximum \SI{235}{\milli\metre}.
\item Page headers and footers must be left empty.
\item No page numbers.
\item Check indentations and spacing by comparing to the example PDF file.
\end{itemize}

\subsubsection{Section headings}

Section headings are centred in boldface with the first word capitalised and the rest of the heading in lower case. Sub-headings appear like major headings, except they start at the left margin in the column. Sub-sub-headings appear like sub-headings, except they are in italics and not boldface. See the examples in this file. No more than 3 levels of headings should be used.

\subsubsection{Fonts}

Times or Times Roman font is used for the main text. Font size in the main text must be 9 points, and in the References section 8 points. Other font types may be used if needed for special purposes. \LaTeX\xspace users should use Adobe Type 1 fonts such as Times or Times Roman, which is done automatically by the provided \LaTeX\xspace class. Do not use Type 3 (bitmap) fonts. Phonemic transcriptions should be placed between forward slashes and phonetic transcriptions between square brackets, for example \textipa{/lO: \ae nd O:d3/} vs. \textipa{[lO:r@nO:d@]}, and authors are encouraged to use the terms `phoneme' and `phone' correctly \cite{moore19_interspeech}.


\subsubsection{Hyperlinks}

For technical reasons, the proceedings editor may strip all active links from the papers during processing. URLs can be included in your paper if written in full, e.g., \url{https://www.interspeech2026.org/}. The text must be all black. Please make sure that they are legible when printed on paper.


\subsection{Figures}\label{sec:figures}

Figures must be centred in the column or page. Figures which span 2 columns must be placed at the top or bottom of a page.
Captions should follow each figure and have the format used in Figure~\ref{fig:speech_production}. Diagrams should preferably be vector graphics. Figures must be legible when printed in monochrome on DIN A4 paper; a minimum font size of  8 points for all text within figures is recommended. Diagrams must not use stipple fill patterns because they will not reproduce properly in Adobe PDF. See \textit{APA Guidelines regarding the use of color in Figures} (\url{https://apastyle.apa.org/style-grammar-guidelines/tables-figures/colors}).

From an accessibility standpoint, please ensure that your paper remains coherent for a reader who cannot see your figures. Please make sure that:
    \begin{itemize}
        \item all figures are referenced in the text.
        \item figures are fully explained either in text, in their caption, or in a combination of the two.
        \item instead of visualizing differences between areas in your figures only by color, those differences can also be illustrated by using different textures or patterns.
        \item all text, including axis labels, is at least as large as the content text surrounding the figures.
        \item \blue{We encourage to use color-blind–friendly color palettes (see \url{https://www.color-blindness.com/coblis-color-blindness-simulator/})}
    \end{itemize}


\subsection{Tables}\label{sec:tables}

An example of a table is shown in Table~\ref{tab:example}. The caption text must be above the table. Tables must be legible when printed in monochrome on DIN A4 paper; a minimum font size of 8 points is recommended. 

From an accessibility standpoint, please make sure that:
    \begin{itemize}
        \item tables are not images.
        \item all table headings are clearly marked as such, and are distinct from table cells.
        \item all tables are referenced in the text.
    \end{itemize}

\begin{table}[th]
  \caption{This is an example of a table}
  \label{tab:example}
  \centering
  \begin{tabular}{ r@{}l  r }
    \toprule
    \multicolumn{2}{c}{\textbf{Ratio}} &
                                         \multicolumn{1}{c}{\textbf{Decibels}} \\
    \midrule
    $1$                       & $/10$ & $-20$~~~             \\
    $1$                       & $/1$  & $0$~~~               \\
    $2$                       & $/1$  & $\approx 6$~~~       \\
    $3.16$                    & $/1$  & $10$~~~              \\
    $10$                      & $/1$  & $20$~~~              \\
    \bottomrule
  \end{tabular}

\end{table}



\subsection{Equations}

Equations should be placed on separate lines and numbered. We define
%
\begin{align}
  x(t) &= s(t') \nonumber \\
       &= s(f_\omega(t))
\end{align}
%
where \(f_\omega(t)\) is a special warping function. Equation \ref{equation:eq2} is a little more complicated.
%
\begin{align}
  f_\omega(t) &= \frac{1}{2 \pi j} \oint_C
  \frac{\nu^{-1k} \mathrm{d} \nu}
  {(1-\beta\nu^{-1})(\nu^{-1}-\beta)}
  \label{equation:eq2}
\end{align}
%



\begin{figure}[t]
  \centering
  \includegraphics[width=\linewidth]{figure.pdf}
  \caption{Schematic diagram of speech production.}
  \label{fig:speech_production}
\end{figure}


\subsection{Style}

Manuscripts must be written in English. Either US or UK spelling is acceptable (but do not mix them).


\subsubsection{International System of Units (SI)}

Use SI units, correctly formatted with a non-breaking space between the quantity and the unit. In \LaTeX\xspace this is best achieved using the \texttt{siunitx} package (which is already included by the provided \LaTeX\xspace class). This will produce
\SI{25}{\milli\second}, \SI{44.1}{\kilo\hertz} and so on.


\begin{table}[b!]
  \caption{Main predefined styles in Word}
  \label{tab:word_styles}
  \centering
  \begin{tabular}{ll}
    \toprule
    \textbf{Style Name}      & \textbf{Entities in a Paper}                \\
    \midrule
    Title                    & Title                                       \\
    Author                   & Author name                                 \\
    Affiliation              & Author affiliation                          \\
    Email                    & Email address                               \\
    AbstractHeading          & Abstract section heading                    \\
    Body Text                & First paragraph in abstract                 \\
    Body Text Next           & Following paragraphs in abstract            \\
    Index                    & Index terms                                 \\
    1. Heading 1             & 1\textsuperscript{st} level section heading \\
    1.1 Heading 2            & 2\textsuperscript{nd} level section heading \\
    1.1.1 Heading 3          & 3\textsuperscript{rd} level section heading \\
    Body Text                & First paragraph in section                  \\
    Body Text Next           & Following paragraphs in section             \\
    Figure Caption           & Figure caption                              \\
    Table Caption            & Table caption                               \\
    Equation                 & Equations                                   \\
    \textbullet\ List Bullet & Bulleted lists                              \\\relax
    [1] Reference            & References                                  \\
    \bottomrule
  \end{tabular}
\end{table}


\section{Specific information for Microsoft Word}

For ease of formatting, please use the styles listed in Table \ref{tab:word_styles}. The styles are defined in the Word version of this template and are shown in the order in which they would be used when writing a paper. When the heading styles in Table \ref{tab:word_styles} are used, section numbers are no longer required to be typed in because they will be automatically numbered by Word. Similarly, reference items will be automatically numbered by Word when the ``Reference'' style is used.

If your Word document contains equations, you must not save your Word document from ``.docx'' to ``.doc'' because this will convert all equations to images of unacceptably low resolution.

\section{Submissions}

Information on how and when to submit your paper is provided on the conference website.

\subsection{Manuscript}

Authors are required to submit a single PDF file of each manuscript. The PDF file should comply with the following requirements: (a) no password protection; (b) all fonts must be embedded; (c) text searchable (do ctrl-F and try to find a common word such as ``the''), and (d) the document properties do not reveal the identity or username. The conference organisers may contact authors of non-complying files to obtain a replacement. Papers for which an acceptable replacement is not provided in a timely manner will be withdrawn.

\subsubsection{Embed all fonts}

It is \textit{very important} that the PDF file embeds all fonts!  PDF files created using \LaTeX, including on \url{https://overleaf.com}, will generally embed all fonts from the body text. However, it is possible that included figures (especially those in PDF or PS format) may use additional fonts that are not embedded, depending how they were created.

On Windows, the bullzip printer can convert any PDF to have embedded and subsetted fonts. On Linux \& MacOS, converting to and from Postscript will embed all fonts:
\\

\noindent\textsf{pdf2ps file.pdf}\\
\noindent\textsf{ps2pdf -dPDFSETTINGS=/prepress file.ps file.pdf}


\subsubsection{Sanitise PDF metadata}
\label{section:pdf_sanitise}

Check that author identity is not revealed in the PDF metadata. The provided \LaTeX\xspace class ensures this. Metadata can be inspected using a PDF viewer.


\subsection{Optional supplementary material or links to online material}
\label{section:multimedia}

\subsubsection{Supplementary material}
\label{section:supplementary}

Interspeech offers the option of submitting supplementary material.
These files are meant for code, data or audio-visual illustrations that cannot be conveyed in text, tables and graphs. Just as with figures used in your manuscript, make sure that you have sufficient author rights to all other materials that you submit. This material will be available to reviewers and committee members for evaluating the paper, but it will not be published with the paper or accessible in the archives. To make material available for future readers, please use online resources (see section \ref{section:repos}). Supplementary material is intended to support the reviewers in assessing the paper; however, the paper should be understandable on its own. Reviewers may choose to examine the supplementary material but are not required to do so.

In line with the double-blind review policy, make sure that the supplementary material is anonymised. Unless strictly necessary, avoid to provide code at this step, which is usually harder to anonymise.

Your supplementary material must be submitted in a single ZIP file limited to 100 MB for each separate paper. Within the ZIP file you can use folders to organise the files. In the ZIP file you should include a \texttt{README.txt} or \texttt{index.html} file to describe the content. In the manuscript, refer to a supplementary file by filename. Use short file names with no spaces.

\subsubsection{Online resources such as web sites, blog posts, code, and data}
\label{section:repos}

It is common for manuscripts to provide links to web sites (e.g., as an alternative to including a ZIP file as supplementary material), code repositories, data sets, or other online resources. Provision of such materials is generally encouraged; however, they should not be used to circumvent the limit on manuscript length. In particular, reviewers will not be required to read any content other than the main manuscript.

Links included in the version submitted for review should not reveal the identity or affiliation of the authors. If this is not possible, then the links should not be provided in the version submitted for review, but they can be added in the final version of the paper, if accepted. A placeholder can be included in the paper with a text like: ``[to ensure author anonymity, the link to the resource will be added after the review process]''. Links and repositories provided for review should be anonymous. In particular, names in repositories are forbidden.

Authors are encouraged to group any supplementary material in a single repository. If the paper is accepted, authors will have the option to include a single link to this repository in the paper's metadata on ISCA Archives. For that purpose, the link must be a DOI link. Free DOI links can for instance be obtained through Zenodo, with additional guidance available in GitHub Docs (\url{https://docs.github.com/en/repositories/archiving-a-github-repository/referencing-and-citing-content})

\section{Discussion}

Authors must proofread their PDF file prior to submission, to ensure it is correct. Do not rely on proofreading the \LaTeX\xspace source or Word document. \textbf{Please proofread the PDF file before it is submitted.}

\section{Acknowledgments}

{\color{blue}Acknowledgments should be included only in the camera-ready version, not in the version submitted for review. For regular papers, pages 5 and 6, and for long papers, pages 9 and 10, are reserved exclusively for acknowledgments, disclosures of the use of generative AI tools, and references. No other content may appear on these pages. Any appendices must be contained within the first four pages for regular papers and within the first eight pages for long papers.

Acknowledgments and references may begin on an earlier page if space permits.}

\ifcameraready
     The Interspeech 2026 organizers
\else
     The authors
\fi
would like to thank ISCA and the organizing committees of past Interspeech conferences for their help and for kindly providing the previous version of this template.




{\color{blue}
\section{Generative AI Use Disclosure}
The extent of Generative AI use must be disclosed. This section may be in the 5th or 6th pages of regular papers, or the 9th or 10th pages of long papers.  ISCA policy says: \textit{All (co-)authors must be responsible and accountable for the work and content of the paper, and they must consent to its submission. Any generative AI tools cannot be a co-author of the paper. They can be used for editing and polishing manuscripts, but should not be used for producing a significant part of the manuscript}.}



\section{Reference Format}

{\color{blue}It is ISCA policy that papers submitted to Interspeech should refer to peer-reviewed publications. References to non-peer-reviewed publications (including public repositories such as arXiv, Preprints, and HAL, software, and personal communications) should only be made if there is no peer-reviewed publication available, and should be kept to a minimum.

References should be in standard IEEE format (follows the IEEEtran format), numbered in order of appearance, for example \cite{Davis80-COP} is cited before \cite{Rabiner89-ATO}. For longer works such as books, provide a single entry for the complete work in the References, then cite specific pages \cite[pp.\ 417--422]{Hastie09-TEO} or a chapter \cite[Chapter 2]{Hastie09-TEO}. Multiple references may be cited in a list \cite{Smith22-XXX, Jones22-XXX}.}


\bibliographystyle{IEEEtran}
\bibliography{mybib}

\end{document}

%%% Local Variables:
%%% mode: LaTeX
%%% TeX-master: t
%%% End: